\chapter{Exemplos de formatação}
\label{ch:ex}

Essa frase é verdadeira pois tem um \mt|\cite| no final \cite{turing1937}. Essa é mais verdadeira ainda pois tem um (\mt|\cite{turing1937,dijkstra1968}|) no final \cite{turing1937,dijkstra1968}. Já esta frase inofensiva usa \mt|\citeonline{dijkstra1968}| para citar \citeonline{dijkstra1968} nominalmente. O trabalho de \citeonline{diffie1976} foi altamente influente \cite{diffie1976}. Essa outra frase cita o trabalho que \citeonline{Saleem2018} escreveu com outros 4 autores.

Mais algumas citações de tipos específicos de documentos:
\begin{itemize}
\item \mt|@inproceedings|. \citeonline{Ullman1989magic}. Jabuti \cite{Ullman1989magic}.
\item \mt|@article|. \citeonline{Distefano2019}. Framboesa \cite{Distefano2019}.
\item \mt|@book|. \citeonline{Abiteboul1995}. Goiaba \cite{Abiteboul1995}.
\item \mt|@incollection| \citeonline{Forgy1989}. Melancia \cite{Forgy1989}.
\item \mt|@techreport| \citeonline{rdf11}. Figo \cite{rdf11}.
\end{itemize}

A lista abaixo mostra o efeito de \mt|\autoref{}| com capítulos e (sub)seções.

\begin{itemize}
\item Há coisas na \autoref{ch:ex}.
\item Há coisas na \autoref{sec:stuff}.
\item Há coisas na \autoref{sec:more}.
\item Há coisas na \autoref{sec:yet-more}.
\end{itemize}

Atenção! O template da BU deixa figuras e tabelas alinhadas à esquerda. No entanto, o tutorial de Word disponibilizado pela BU diz que Legendas e captions devem respeitar o ``alinhamento da ilustração'' (e apresenta uma ilustração alinhada a esquerda). O tutorial explicando a ABNT mostra uma figura centralizada com legendas alinhadas a esquerda e com recuo até o começo da figura. O autor do .cls se exime de qualquer culpa. Alinhe aqui (com \mt|\centering|, \mt|\flushright| ou \mt|\flushleft|) como mandar o seu coração.

A \autoref{fig:figura} é uma figura de exemplo. Para gerar o ``Fonte:'' (que aparece como ``Source:'' na versão inglês') foi usada o comando \mt|\captionsource{o autor}|. Essa macro, \mt|\captionsource| foi introduzida pela classe \texttt{ufsc-thesis-rn46-2019}.

\begin{figure}[t]
  \centering
  \caption{\footnotesize A caption text.}
  \label{fig:figura}
  %\lipsum[1]

  \includegraphics[width=.5\linewidth]{../logo-ufsc.pdf}
  \captionsource{The author.}
\end{figure}

\section{Coisas}
\label{sec:stuff}
Imagine alguma afirmação de alto valor científico aqui.

\subsection{Outras coisas}
\label{sec:more}
Olá! Eu vim do passado para te avisar que o texto de uma dissertação deve ser impessoal e você não deveria tentar conversar com o leitor. 

\subsubsection{Outras coisas mais}
\label{sec:yet-more}
Estudos demonstram que essa afirmação é falsa.

\subsubsubsection{Ainda outras coisas mais}
\label{sec:yet-another}
Fazer a grama verde, como? Novamente o jogo foi perdido. Opcionalmente, tudo pode ser opcional. Recursos foram gastos com isso. Descubra a verdade nas capitalizadas.
% Fiquei 15 minutos mais próximo da morte ao escrever isso. Você pode chegar ainda mais perto se tentar entender.
