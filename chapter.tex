Essa frase é verdadeira pois tem um \mt|\cite| no final \cite{turing1937}. Essa
é mais verdadeira ainda pois tem um (\mt|\cite{turing1937,dijkstra1968}|) no
final \cite{turing1937,dijkstra1968}. Já esta frase inofensiva usa
\mt|\citeonline{dijkstra1968}| para citar \citeonline{dijkstra1968}
nominalmente. O trabalho de \citeonline{diffie1976} foi altamente influente
\cite{diffie1976}. Essa outra frase cita o trabalho que \citeonline{Saleem2018}
escreveu com outros 4 autores.

Mais algumas citações de tipos específicos de documentos:
\begin{itemize}
\item \mt|@inproceedings|. \citeonline{Ullman1989magic}. Jabuti
  \cite{Ullman1989magic}.
\item \mt|@article|. \citeonline{Distefano2019}. Framboesa \cite{Distefano2019}.
\item \mt|@book|. \citeonline{Abiteboul1995}. Goiaba \cite{Abiteboul1995}.
\item \mt|@incollection| \citeonline{Forgy1989}. Melancia \cite{Forgy1989}.
\item \mt|@techreport| \citeonline{rdf11}. Figo \cite{rdf11}.
\end{itemize}

A lista abaixo mostra o efeito de \mt|\autoref{}| com capítulos e (sub)seções.

\begin{itemize}
\item Há coisas na \autoref{ch:ex}.
\item Há coisas na \autoref{sec:stuff}.
\item Há coisas na \autoref{sec:more}.
\item Há coisas na \autoref{sec:yet-more}.
\end{itemize}

Citações são feitas com \mt|\begin{citacao}...\end{citacao}|. A BU faz
as mesmas exigências que já são o \textit{default} na classe \abnTeX2.

\begin{citacao}
  A elaboração do trabalho de conclusão de curso em nível de mestrado
  e de doutorado na UFSC deverá atender aos critérios e procedimentos
  estabelecidos nesta resolução normativa e em diretrizes
  estabelecidas pela Pró-Reitoria de Pós-Graduação e pelos Programas
  de Pós-Graduação.
\end{citacao}

Atenção! O template da BU deixa figuras e tabelas alinhadas à esquerda. No
entanto, o tutorial de Word disponibilizado pela BU diz que legendas e
\emph{captions} devem respeitar o ``alinhamento da ilustração'' (e apresenta
uma ilustração alinhada à esquerda). O tutorial explicando a ABNT mostra uma
figura centralizada com legendas alinhadas a esquerda e com recuo até o começo
da figura. O autor do \texttt{.cls} se exime de qualquer culpa. Alinhe aqui
(com \mt|\centering|, \mt|\flushright| ou \mt|\flushleft|) como mandar o seu
coração. Veja na \autoref{fig:logo} o efeito de se usar \mt|\centering|.

\begin{figure}[t]
  \centering
  \caption{Logotipo da Universidade Federal de Santa Catarina.}
  \label{fig:logo}

  \includegraphics[width=.2\linewidth]{../logo-ufsc.pdf}
  \captionsource{O autor.}
\end{figure}

\section{Coisas}
\label{sec:stuff}
Imagine alguma afirmação de alto valor científico aqui.

\subsection{Outras coisas}
\label{sec:more}
Olá! Eu vim do passado para te avisar que o texto de uma dissertação deve ser
impessoal e você não deveria tentar conversar com o leitor.

\subsubsection{Outras coisas mais}
\label{sec:yet-more}
Estudos demonstram que essa afirmação é falsa.

\subsubsubsection{Ainda outras coisas mais}
\label{sec:yet-another}
Fazer a grama verde, como? Novamente o jogo foi perdido. Opcionalmente, tudo
pode ser opcional. Recursos foram gastos com isso. Descubra a verdade nas
capitalizadas.
% Fiquei 15 minutos mais próximo da morte ao escrever isso. Você pode chegar
% ainda mais perto se tentar entender.
