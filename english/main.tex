\documentclass[english,embeddedlogo]{../ufsc-thesis-rn46-2019/ufsc-thesis-rn46-2019}
\usepackage[utf8]{inputenc} % UTF-8

% Usado para mostrar código
\usepackage{minted}
\newmintinline[mt]{latex}{fontsize=\normalsize}
\setminted{fontsize=\tiny,linenos,xleftmargin=2em}
\setmintedinline{breaklines,breakbytokenanywhere}


%%%%%%%%%%%%%%%%%%%%%%%%%%%%%%%%%%%%%%%%%%%%%%%%%%%%%%%%%%%%%%%%%%%%
%%% Configurações da classe (dados do trabalho)                  %%%
%%%%%%%%%%%%%%%%%%%%%%%%%%%%%%%%%%%%%%%%%%%%%%%%%%%%%%%%%%%%%%%%%%%%

% Preâmbulo
\titulo{Template \LaTeX~ seguindo a RN 46/2019/CPG da UFSC}
\autor{Omar Ravenhurst}
% Importante! Para documentos em inglês, não use today, digite a data em
% pt_BR, como deve aparecer na folha de certificação.
\data{1 de Agosto de 2019}
\instituicao{Universidade Federal de Santa Catarina}
\programa{Programa de Pós-Graduação em Ciências da Computação}
\tese % ou \dissertacao
\local{Florianópolis} % Apenas cidade! Sem estado
\titulode{Doutor em Ciência da Computação}
\orientador{Prof. Dr. Ben Trovato}
\coorientador{Prof. Dr. Lars Thørväld}
\centro{Centro Tecnológico -- CTC}

% Membros da banca e coordenador
% As regras da BU agora exigem que Dr. apareça depois do nome
\membrobanca{Prof. Valerie Béranger, Dr.}{Universidade Federal de Santa Catarina}
\membrobanca{Prof. Mordecai Malignatus, Dr.}{Universidade Federal de Santa Catarina}
\membrobanca{Prof. Huifen Chan, Dr.}{Universidade Federal de Santa Catarina}
% Atenção! o template da BU e o documento que apresenta as regras continua
% usando Dr antes do nome para Orientador e Coordenador!
\coordenador{Prof. Dr. Charles Palmer, Dr.}


\begin{document}

%%%%%%%%%%%%%%%%%%%%%%%%%%%%%%%%%%%%%%%%%%%%%%%%%%%%%%%%%%%%%%%%%%%%
%%% Principais elementos pré-textuais                            %%%
%%%%%%%%%%%%%%%%%%%%%%%%%%%%%%%%%%%%%%%%%%%%%%%%%%%%%%%%%%%%%%%%%%%%

% Inicia parte pré-textual do documento capa, folha de rosto, folha de
% aprovação, aprovação, resumo, lista de tabelas, lista de figuras, etc.
\pretextual%
\imprimircapa%
\imprimirfolhaderosto*
\protect\incluirfichacatalografica{../ficha.pdf}
\imprimirfolhadecertificacao
\clearpage \listoffigures* 
\clearpage \tableofcontents*%

%%%%%%%%%%%%%%%%%%%%%%%%%%%%%%%%%%%%%%%%%%%%%%%%%%%%%%%%%%%%%%%%%%%%
%%% Corpo do texto                                               %%%
%%%%%%%%%%%%%%%%%%%%%%%%%%%%%%%%%%%%%%%%%%%%%%%%%%%%%%%%%%%%%%%%%%%%
\textual%

\chapter{Exemplos de formatação}
\label{ch:ex}

Essa frase é verdadeira pois tem um \mt|\cite| no final \cite{turing1937}. Essa é mais verdadeira ainda pois tem um (\mt|\cite{turing1937,dijkstra1968}|) no final \cite{turing1937,dijkstra1968}. Já esta frase inofensiva usa \mt|\citeonline{dijkstra1968}| para citar \citeonline{dijkstra1968} nominalmente. O trabalho de \citeonline{diffie1976} foi altamente influente \cite{diffie1976}. Essa outra frase cita o trabalho que \citeonline{golub1992} escreveu com outros 9 autores.


A lista abaixo mostra o efeito de \mt|\autoref{}| com capítulos e (sub)seções.

\begin{itemize}
\item \autoref{ch:ex}
\item \autoref{sec:stuff}
\item \autoref{sec:more}
\item \autoref{sec:yet-more}
\end{itemize}

Atenção! O template da BU deixa figuras e tabelas alinhadas à esquerda. No entanto, o tutorial de Word disponibilizado pela BU diz que Legendas e captions devem respeitar o ``alinhamento da ilustração'' (e apresenta uma ilustração alinhada a esquerda). O tutorial explicando a ABNT mostra uma figura centralizada com legendas alinhadas a esquerda e com recuo até o começo da figura. O autor do .cls se exime de qualquer culpa. Alinhe aqui (com \mt|\centering|, \mt|\flushright| ou \mt|\flushleft|) como mandar o seu coração.

A \autoref{fig:figura} é uma figura de exemplo. Para gerar o ``Fonte:'' (que aparece como ``Source:'' na versão inglês') foi usada o comando \mt|\captionsource{o autor}|. Essa macro, \mt|\captionsource| foi introduzida pela classe \texttt{ufsc-thesis-rn46-2019}.

\begin{figure}[t]
  \centering
  \caption{\footnotesize A caption text.}
  \label{fig:figura}
  %\lipsum[1]

  \includegraphics[width=.5\linewidth]{../logo-ufsc.pdf}
  \captionsource{The author.}
\end{figure}

\section{Coisas}
\label{sec:stuff}
Imagine alguma afirmação de alto valor científico aqui.

\subsection{Outras coisas}
\label{sec:more}
Olá! Eu vim do passado para te avisar que o texto de uma dissertação deve ser impessoal e você não deveria tentar conversar com o leitor. 

\subsubsection{Outras coisas mais}
\label{sec:yet-more}
Estudos demonstram que essa afirmação é falsa.

\subsubsubsection{Ainda outras coisas mais}
\label{sec:yet-another}
Fazer a grama verde, como? Novamente o jogo foi perdido. Opcionalmente, tudo pode ser opcional. Recursos foram gastos com isso. Descubra a verdade nas capitalizadas.
% Fiquei 15 minutos mais próximo da morte ao escrever isso. Você pode chegar ainda mais perto se tentar entender.

%%%%%%%%%%%%%%%%%%%%%%%%%%%%%%%%%%%%%%%%%%%%%%%%%%%%%%%%%%%%%%%%%%%%
%%% Elementos pós-textuais                                       %%%
%%%%%%%%%%%%%%%%%%%%%%%%%%%%%%%%%%%%%%%%%%%%%%%%%%%%%%%%%%%%%%%%%%%%

\postextual
\bibliography{main}

\end{document}
